\documentclass{article}\usepackage[]{graphicx}\usepackage[]{xcolor}
% maxwidth is the original width if it is less than linewidth
% otherwise use linewidth (to make sure the graphics do not exceed the margin)
\makeatletter
\def\maxwidth{ %
  \ifdim\Gin@nat@width>\linewidth
    \linewidth
  \else
    \Gin@nat@width
  \fi
}
\makeatother

\definecolor{fgcolor}{rgb}{0.345, 0.345, 0.345}
\newcommand{\hlnum}[1]{\textcolor[rgb]{0.686,0.059,0.569}{#1}}%
\newcommand{\hlsng}[1]{\textcolor[rgb]{0.192,0.494,0.8}{#1}}%
\newcommand{\hlcom}[1]{\textcolor[rgb]{0.678,0.584,0.686}{\textit{#1}}}%
\newcommand{\hlopt}[1]{\textcolor[rgb]{0,0,0}{#1}}%
\newcommand{\hldef}[1]{\textcolor[rgb]{0.345,0.345,0.345}{#1}}%
\newcommand{\hlkwa}[1]{\textcolor[rgb]{0.161,0.373,0.58}{\textbf{#1}}}%
\newcommand{\hlkwb}[1]{\textcolor[rgb]{0.69,0.353,0.396}{#1}}%
\newcommand{\hlkwc}[1]{\textcolor[rgb]{0.333,0.667,0.333}{#1}}%
\newcommand{\hlkwd}[1]{\textcolor[rgb]{0.737,0.353,0.396}{\textbf{#1}}}%
\let\hlipl\hlkwb

\usepackage{framed}
\makeatletter
\newenvironment{kframe}{%
 \def\at@end@of@kframe{}%
 \ifinner\ifhmode%
  \def\at@end@of@kframe{\end{minipage}}%
  \begin{minipage}{\columnwidth}%
 \fi\fi%
 \def\FrameCommand##1{\hskip\@totalleftmargin \hskip-\fboxsep
 \colorbox{shadecolor}{##1}\hskip-\fboxsep
     % There is no \\@totalrightmargin, so:
     \hskip-\linewidth \hskip-\@totalleftmargin \hskip\columnwidth}%
 \MakeFramed {\advance\hsize-\width
   \@totalleftmargin\z@ \linewidth\hsize
   \@setminipage}}%
 {\par\unskip\endMakeFramed%
 \at@end@of@kframe}
\makeatother

\definecolor{shadecolor}{rgb}{.97, .97, .97}
\definecolor{messagecolor}{rgb}{0, 0, 0}
\definecolor{warningcolor}{rgb}{1, 0, 1}
\definecolor{errorcolor}{rgb}{1, 0, 0}
\newenvironment{knitrout}{}{} % an empty environment to be redefined in TeX

\usepackage{alltt}
\usepackage[margin=1.0in]{geometry} % To set margins
\usepackage{amsmath}  % This allows me to use the align functionality.
                      % If you find yourself trying to replicate
                      % something you found online, ensure you're
                      % loading the necessary packages!
\usepackage{amsfonts} % Math font
\usepackage{fancyvrb}
\usepackage{hyperref} % For including hyperlinks
\usepackage[shortlabels]{enumitem}% For enumerated lists with labels specified
                                  % We had to run tlmgr_install("enumitem") in R
\usepackage{float}    % For telling R where to put a table/figure
\usepackage{natbib}        %For the bibliography
\bibliographystyle{apalike}%For the bibliography
\IfFileExists{upquote.sty}{\usepackage{upquote}}{}
\begin{document}

\begin{enumerate}
%%%%%%%%%%%%%%%%%%%%%%%%%%%%%%%%%%%%%%%%%%%%%%%%%%%%%%%%%%%%%%%%%%%%%%%%%%%%%%%%
%%%%%%%%%%%%%%%%%%%%%%%%%%%%%%%%%%%%%%%%%%%%%%%%%%%%%%%%%%%%%%%%%%%%%%%%%%%%%%%%
% QUESTION 1
%%%%%%%%%%%%%%%%%%%%%%%%%%%%%%%%%%%%%%%%%%%%%%%%%%%%%%%%%%%%%%%%%%%%%%%%%%%%%%%%
%%%%%%%%%%%%%%%%%%%%%%%%%%%%%%%%%%%%%%%%%%%%%%%%%%%%%%%%%%%%%%%%%%%%%%%%%%%%%%%%
\item This week's Problem of the Week in Math is described as follows:
\begin{quotation}
  \textit{There are thirty positive integers less than 100 that share a certain 
  property. Your friend, Blake, wrote them down in the table to the left. But 
  Blake made a mistake! One of the numbers listed is wrong and should be replaced 
  with another. Which number is incorrect, what should it be replaced with, and 
  why?}
\end{quotation}
The numbers are listed below.
\begin{center}
  \begin{tabular}{ccccc}
    6 & 10 & 14 & 15 & 21\\
    22 & 26 & 33 & 34 & 35\\
    38 & 39 & 46 & 51 & 55\\
    57 & 58 & 62 & 65 & 69\\
    75 & 77 & 82 & 85 & 86\\
    87 & 91 & 93 & 94 & 95
  \end{tabular}
\end{center}
Use the fact that the ``certain'' property is that these numbers are all supposed
to be the product of \emph{unique} prime numbers to find and fix the mistake that
Blake made.\\
\textbf{Reminder:} Code your solution in an \texttt{R} script and copy it over
to this \texttt{.Rnw} file.\\
\textbf{Hint:} You may find the \verb|%in%| operator and the \verb|setdiff()| function to be helpful.\\

\textbf{Solution:} 
% Write your answer and explanations here.

\begin{knitrout}\scriptsize
\definecolor{shadecolor}{rgb}{0.969, 0.969, 0.969}\color{fgcolor}\begin{kframe}
\begin{alltt}
\hldef{prime_factors} \hlkwb{=} \hlkwa{function}\hldef{(}\hlkwc{n}\hldef{)\{} \hlcom{#returns a vector of prime factors of n}
  \hldef{primefactors} \hlkwb{=} \hlkwd{c}\hldef{()} \hlcom{#stores prime factors of n}
  \hldef{i} \hlkwb{=} \hlnum{2} \hlcom{#initialize i to 2 because everything is divisible by 1, and we don't care about that}
  \hlkwa{while}\hldef{(n}\hlopt{>}\hlnum{1}\hldef{)\{}
    \hlkwa{while}\hldef{(n} \hlopt \hldef{i} \hlopt{==} \hlnum{0}\hldef{)\{} \hlcom{#if i is a factor}
      \hldef{n} \hlkwb{=} \hldef{n}\hlopt{/}\hldef{i} \hlcom{#divides by factor to potentially get other factors of n}
      \hldef{primefactors} \hlkwb{=} \hlkwd{append}\hldef{(primefactors, i)} \hlcom{#add i to vector of prime factors}
    \hldef{\}}
    \hldef{i}\hlkwb{=}\hldef{i}\hlopt{+}\hlnum{1} \hlcom{#find next factor}
    \hlkwa{if}\hldef{(i}\hlopt{*}\hldef{i}\hlopt{>}\hldef{n)\{} \hlcom{#checks that n has no more prime divisors}
      \hlkwa{if}\hldef{(n}\hlopt{>}\hlnum{1}\hldef{)\{} \hlcom{#if n=1, we can break}
        \hldef{primefactors} \hlkwb{=} \hlkwd{append}\hldef{(primefactors, n)} \hlcom{#append the prime n to vector of prime factors}
      \hldef{\}}
      \hlkwa{break} \hlcom{#if n=1, we are done}
    \hldef{\}}
  \hldef{\}}
  \hlkwd{return}\hldef{(primefactors)} \hlcom{#return vector of all prime factors (includes duplicates)}
\hldef{\}}


\hldef{is_prime} \hlkwb{=} \hlkwa{function}\hldef{(}\hlkwc{n}\hldef{)\{} \hlcom{#checks if n is prime}
  \hlkwa{for} \hldef{(i} \hlkwa{in} \hlnum{3}\hlopt{:}\hldef{n}\hlopt{-}\hlnum{1}\hldef{)\{}
    \hlkwa{if}\hldef{(n}\hlopt\hldef{i} \hlopt{==} \hlnum{0}\hldef{)\{} \hlcom{#checks if divisible by i}
      \hlkwd{return}\hldef{(}\hlnum{FALSE}\hldef{)}
    \hldef{\}}
  \hldef{\}}
  \hlkwd{return} \hldef{(}\hlnum{TRUE}\hldef{)} \hlcom{#not divisible by anything other than itself and 1}
\hldef{\}}

\hldef{correct} \hlkwb{=} \hlkwd{c}\hldef{()} \hlcom{#empty vector that will be a vector of products of unique prime numbers}
\hlkwa{for}\hldef{(i} \hlkwa{in} \hlnum{3}\hlopt{:}\hlnum{99}\hldef{)\{}
  \hldef{primefactors} \hlkwb{=} \hlkwd{prime_factors}\hldef{(i)} \hlcom{#gets prime factors of i}
  \hlkwa{if}\hldef{(}\hlkwd{length}\hldef{(primefactors)}\hlopt{==}\hlkwd{length}\hldef{(}\hlkwd{unique}\hldef{(primefactors))}  \hlopt{&&} \hlkwd{length}\hldef{(primefactors)} \hlopt{==} \hlnum{2}\hldef{)\{} \hlcom{#checks if they are all unique}
    \hlkwa{if}\hldef{(}\hlopt{!}\hlkwd{is_prime}\hldef{(i))\{} \hlcom{#can't be prime because it must be a product of unique prime numbers}
      \hldef{correct} \hlkwb{=} \hlkwd{append}\hldef{(correct, i)} \hlcom{#add i if so}
    \hldef{\}}
  \hldef{\}}
\hldef{\}}

\hldef{nums} \hlkwb{=} \hlkwd{c}\hldef{(}\hlnum{6}\hldef{,} \hlnum{10}\hldef{,} \hlnum{14}\hldef{,} \hlnum{15}\hldef{,} \hlnum{21}\hldef{,} \hlnum{22}\hldef{,} \hlnum{26}\hldef{,} \hlnum{33}\hldef{,} \hlnum{34}\hldef{,} \hlnum{35}\hldef{,} \hlnum{38}\hldef{,} \hlnum{39}\hldef{,} \hlnum{46}\hldef{,} \hlnum{51}\hldef{,} \hlnum{55}\hldef{,} \hlnum{57}\hldef{,} \hlnum{58}\hldef{,}
         \hlnum{62}\hldef{,} \hlnum{65}\hldef{,} \hlnum{69}\hldef{,} \hlnum{75}\hldef{,} \hlnum{77}\hldef{,} \hlnum{82}\hldef{,} \hlnum{85}\hldef{,} \hlnum{86}\hldef{,} \hlnum{87}\hldef{,} \hlnum{91}\hldef{,} \hlnum{93}\hldef{,} \hlnum{94}\hldef{,} \hlnum{95}\hldef{)} \hlcom{#set of numbers Blake wrote down}

\hldef{incorrect} \hlkwb{=} \hlkwd{setdiff}\hldef{(nums, correct)} \hlcom{#number that is in Blake's set but not my correct numbers vector}
\hldef{replacement} \hlkwb{=} \hlkwd{setdiff}\hldef{(correct, nums)} \hlcom{#number that is in my correct vector but not Blake's set of numbers}

\hlkwd{paste}\hldef{(}\hlsng{"The incorrect number is "}\hldef{, incorrect,} \hlsng{". Replace it with "}\hldef{, replacement,} \hlkwc{sep} \hldef{=} \hlsng{""}\hldef{)}
\end{alltt}
\begin{verbatim}
## [1] "The incorrect number is 75. Replace it with 74"
\end{verbatim}
\end{kframe}
\end{knitrout}
\end{enumerate}

\bibliography{bibliography}
\end{document}
